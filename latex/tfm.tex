\documentclass[12pt, a4paper]{article}

% Packages for formatting
\usepackage[utf8]{inputenc} % UTF-8 encoding
\usepackage[margin=1in]{geometry} % 1-inch margins
\usepackage{times} % Times New Roman font
\usepackage{setspace} % For double spacing
\usepackage[american]{babel}
\usepackage{fancyhdr} % For headers and footers
\usepackage{titlesec} % For section numbering
\usepackage[T1]{fontenc} % Font encoding
\usepackage{csquotes}
\usepackage{caption}
\usepackage{booktabs} % For professional-looking tables
\usepackage[style=apa, backend=biber]{biblatex} % APA 7th edition references
% \addbibresource{references.bib} % Reference file

% Set double spacing
\doublespacing

% Indent paragraphs by 0.5 inch
\setlength{\parindent}{0.5in}

% Page numbering setup
\pagestyle{fancy}
\fancyhf{}
\rfoot{\thepage} % Page number in the bottom-right corner
\renewcommand{\headrulewidth}{0pt} % Remove the header rule

% Section formatting
\titleformat{\section}[block]{\normalfont\Large\bfseries}{\thesection.}{0.5em}{}

\begin{document}
\pagenumbering{roman}
\section*{Abstract} % maximum 150 words
.

\section*{Keywords} % maximum 10 keywords
.

\newpage
\tableofcontents
\newpage

\pagenumbering{arabic} % Switch to Arabic numerals for the main content

\section{Introduction}
.

\section{Methodology}
To what extent can morphological inflection be automatically distinguished from derivation? 

This project will be done using a distributional semantics model (FastText). It focuses on two morphologically rich languages, Spanish and Polish and will examin differences in semantic regularity alsmo among various types of inflection/derivation (aspect).

% \printbibliography

\end{document}
